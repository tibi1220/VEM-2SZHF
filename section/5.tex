\section{Az (1)-es elemben tárolt rugalmas alakváltozási energia}

Egy tetszőleges elemben ébredő rugalmas alakváltozási energia az alábbi
képlettel határozható meg:
\begin{myframe}
  \begin{equation}
    U_i = \frac{1}{2} \rvec U_i^{\mathsf T} \rmat K_i \rvec U_i
    \text.
  \end{equation}
\end{myframe}
A képletben $\rvec U_i$ a lokális elmozdulás-vektort, $\rmat K_i$ pedig az
elemhez tartozó merevségi mátrixot jelöli. Az (1)-es elemben felhalmozódó
rugalmas alakváltozási energia ezek alapján:
\begin{myframe}
  \begin{equation}
    U_1
    = \frac{1}{2} \rvec U_1^{\mathsf T} \rmat K_1 \rvec U_1
    = \pyc{prin_TeX(V["E_i"][0], "J", 4)}
    \text.
  \end{equation}
\end{myframe}
