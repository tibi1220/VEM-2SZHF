\section{A (2)-es elem súlypontjában ébredő feszültségek}

Egy elem tetszőleges pontjában ébredő feszültségek az alábbi képlettel
számítható:
\begin{myframe}
  \begin{equation}
    \rvec \sigma_i(\xi; \eta)
    = \rmat D \, \rmat B_i(\xi; \eta) \rvec U_i
    \text.
  \end{equation}
\end{myframe}
Mivel a (2)-es elem súlypontjában vagyunk kíváncsiak a feszültségek értékeire,
ezért először meg kell határoznunk ennek koordinátáit. Ezek a globális
koordináta-rendszerben az alábbi formulával határozhatóak meg:
\begin{myframe}
  \begin{equation}
    x_s = \frac{1}{6A}\sum _{i=1}^{4}(x_{i}+x_{i+1})(x_{i}\ y_{i+1}-x_{i+1}\ y_{i})
    \text,
    \qquad
    y_s = \frac{1}{6A}\sum _{i=1}^{4}(y_{i}+y_{i+1})(x_{i}\ y_{i+1}-x_{i+1}\ y_{i})
    \text,
  \end{equation}
\end{myframe}
ahol $A_i$ a négyszög előjeles területe, értéke az alábbi képlettel számítható:
\begin{myframe}
  \begin{equation}
    A_i = \frac{1}{2}\sum _{i=1}^{4}(x_{i}\ y_{i+1}-x_{i+1}\ y_{i})
  \end{equation}
\end{myframe}
A fenti kifejezésekbe $i=4$ esetén adódó $i + 1 = 5$-ödik elemek helyére az
első elemhez tartozó mennyiségeket kell behelyettesíteni. A (2)-es elem
súlypontja a globális koordináta rendszerben:
\begin{myframe}
  \begin{equation}
    x_s = \pyc{prin_TeX(V["C"][1]["x"], "mm", 4)}
    \text,
    \qquad
    y_s = \pyc{prin_TeX(V["C"][1]["y"], "mm", 4)}
    \text.
  \end{equation}
\end{myframe}
A (\ref{eq:alma})-es egyenlet segítségével a súlypont koordinátái a lokális
koordináta-rendszerben:
\begin{myframe}
  \begin{equation}
    \sisetup{drop-zero-decimal}
    \xi_s = \pyc{prin_TeX(V["C"][1]["xi"], "", 4)}
    \text,
    \qquad
    \eta_s = \pyc{prin_TeX(V["C"][1]["eta"], "", 4)}
    \text.
  \end{equation}
\end{myframe}
A (2)-es elem súlypontjában ébredő feszültségek tehát:
\begin{myframe}
  \begin{equation}
    \rvec \sigma_2(\xi_s; \eta_s)
    =
    \begin{bmatrix}
      \sigma_x \\ \sigma_y \\ \tau_{xy}
    \end{bmatrix}
    =
    \siplaces{4}
    \begin{bmatrix}
      \pyc{print_matrix(V["sigma_i"][1])}
    \end{bmatrix}
    \mathrm{MPa}
    \text.
  \end{equation}
\end{myframe}
