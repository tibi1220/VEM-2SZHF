\section{Csomóponti eredő elmozdulások}

\subsection{Általános adatok}

A csomópontok koordinátáit az \ref{table:U}. táblázat tartalmazza.
\begin{table}[H]
  \def\arraystretch{1.1}
  \centering
  \caption{A csomópontok koordinátái}
  % \begin{noindent}
  \begin{python}
print(r"\def\arraystretch{1.1}")
print(r"\begin{tabular}{| >{\cellcolor{gray!10}}{c} *{2}{|| X{1.5cm} | X{1.5cm}} |}")
print(r"\hline \rowcolor{yellow!10}")
print(r"Csp. & x & y & x \; [\text{mm}] & y \; [\text{mm}] \\")
print(r"\hline \hline")
for i in range(1, 9):
    print(
        f"{i} & {P[i]['x']} & {P[i]['y']} & " +
        f"{N[i]['x']:.0f} & {N[i]['y']:.0f} \\\\ \hline"
    )
print(r"\end{tabular}")
  \end{python}
  % \end{noindent}
  \label{table:U}
\end{table}

Az egyes elemekhez tartozó csomópontokat a \ref{table:lok}. táblázat foglalja
össze.
\begin{table}[H]
  \def\arraystretch{1.1}
  \centering
  \caption{Elem -- csomópont összerendelések}
  \begin{tabular}{| >{\cellcolor{gray!10}}{c} || c | c | c | c |}
    \hline
    \rowcolor{yellow!10}
    Elem & 1. csp                     & 2. csp                     & 3. csp                     & 4. csp                     \\ \hline \hline
    1    & \py{P["rectangles"][0][0]} & \py{P["rectangles"][0][1]} & \py{P["rectangles"][0][2]} & \py{P["rectangles"][0][3]} \\ \hline
    2    & \py{P["rectangles"][1][0]} & \py{P["rectangles"][1][1]} & \py{P["rectangles"][1][2]} & \py{P["rectangles"][1][3]} \\ \hline
    3    & \py{P["rectangles"][2][0]} & \py{P["rectangles"][2][1]} & \py{P["rectangles"][2][2]} & \py{P["rectangles"][2][3]} \\ \hline
  \end{tabular}
  \label{table:lok}
\end{table}

\subsection{Koordináta-transzformáció}

Mivel elemeink nem szabályos négyszögek, ezért a számítás megkönnyítéséhez
érdemes koordináta-transzformációt végrehajtani. Elemenként egy olyan lokális
$\xi$-$\eta$ koordináta-rendszerbe fogunk áttérni, ahol a lemezek csúcsainak
koordinátái rendre $(-1;-1)$, $(1;-1)$, $(1;1)$, $(-1;1)$. A két
koordináta-reprezentáció közötti kapcsolat:
\begin{myframe}
  \begin{equation}
    x = \sum_{i=1}^4 N_i \cdot x_i
    \text,
    \qquad
    \text{ és }
    \qquad
    y = \sum_{i=1}^4 N_i \cdot y_i
    \text,
  \end{equation}
\end{myframe}
ahol: $(x_i; y_i)$ az egyes csomópontok $x$-$y$ koordináta-rendszerbeli
koordinátája, $N_i$ pedig az $i$-edik formafüggvényt jelöli, melyek alakjai:
\begin{myframe}
  \begin{gather}
    N_1 = \frac{1}{4} (1 - \xi) (1 - \eta) \text, \\
    N_2 = \frac{1}{4} (1 + \xi) (1 - \eta) \text, \\
    N_3 = \frac{1}{4} (1 + \xi) (1 + \eta) \text, \\
    N_4 = \frac{1}{4} (1 - \xi) (1 + \eta) \text.
  \end{gather}
\end{myframe}
Ezek alapján felírható az egyes elemekhez tartozó lokális és a globális
koordináta-rendszerek közötti transzformáció:
\begin{myframe}
  \begin{alignat}{9}
     & (1) \qquad & \rightarrow \qquad x & = \py{my_latex(V["transform"][0]["x"],2)} \text, &  & \qquad y = \py{my_latex(V["transform"][0]["y"],2)}
    \\
     & (2) \qquad & \rightarrow \qquad x & = \py{my_latex(V["transform"][1]["x"],2)} \text, &  & \qquad y = \py{my_latex(V["transform"][1]["y"],2)}
    \\
     & (3) \qquad & \rightarrow \qquad x & = \py{my_latex(V["transform"][2]["x"],2)} \text, &  & \qquad y = \py{my_latex(V["transform"][2]["y"],2)}
  \end{alignat}
\end{myframe}
A leképezés inverze pedig:
\begin{myframe}
  \begin{alignat}{9}
     & (1) \qquad & \rightarrow \qquad \xi & = \py{my_latex(V["transform"][0]["xi"], 4)} \text, &  & \qquad \eta = \py{my_latex(V["transform"][0]["eta"], 4)}
    \\
     & (2) \qquad & \rightarrow \qquad \xi & = \py{my_latex(V["transform"][1]["xi"], 4)} \text, &  & \qquad \eta = \py{my_latex(V["transform"][1]["eta"], 4)}
    \label{eq:alma}
    \\
     & (3) \qquad & \rightarrow \qquad \xi & = \py{my_latex(V["transform"][2]["xi"], 4)} \text, &  & \qquad \eta = \py{my_latex(V["transform"][2]["eta"], 4)}
  \end{alignat}
\end{myframe}

\subsection{A Jacobi-mátrixok}

Írjuk fel az egyes elemekhez tartozó leképezések Jacobi-mátrixát. Ehhez először
A formafüggvények $\xi$ és $\eta$ szerinti parciális deriváltakra van szükségünk:
\begin{myframe}
  \begin{alignat}{9}
    \pdv{N_1}{\xi}  & = \frac{\eta - 1}{4}
    \text, \qquad \quad
    \pdv{N_2}{\xi}  & = \frac{-\eta + 1}{4}
    \text, \qquad \quad
    \pdv{N_3}{\xi}  & = \frac{\eta + 1}{4}
    \text, \qquad \quad
    \pdv{N_4}{\xi}  & = \frac{-\eta - 1}{4}
    \text,                                  \\[3mm]
    \pdv{N_1}{\eta} & = \frac{\xi - 1}{4}
    \text, \qquad \quad
    \pdv{N_2}{\eta} & = \frac{-\xi - 1}{4}
    \text, \qquad \quad
    \pdv{N_3}{\eta} & = \frac{\xi + 1}{4}
    \text, \qquad \quad
    \pdv{N_4}{\eta} & = \frac{-\xi + 1}{4}
    \text.
  \end{alignat}
\end{myframe}
A Jacobi mátrix alakja paraméteresen:
\begin{myframe}
  \begin{equation}
    \rmat J_i = \begin{bmatrix}
      \displaystyle \pdv{x}{\xi}  &
      \displaystyle \pdv{y}{\xi}    \\[4mm]
      \displaystyle \pdv{x}{\eta} &
      \displaystyle \pdv{y}{\eta}   \\
    \end{bmatrix} = \begin{bmatrix}
      \displaystyle \sum \pdv{N_i}{\xi} x_i  &
      \displaystyle \sum \pdv{N_i}{\xi} y_i    \\[4mm]
      \displaystyle \sum \pdv{N_i}{\eta} x_i &
      \displaystyle \sum \pdv{N_i}{\eta} y_i   \\
    \end{bmatrix}
    \text.
  \end{equation}
\end{myframe}
Ezek alapján az egyes elemekhez tartozó Jacobi-mátrixok numerikus alakja:
\begin{myframe}
  \begin{align}
    \rmat J_1 & =
    \py{my_latex(V["J"][0], -1, mat_str="array").replace('cc', "*{2}{X{3cm}}")}
    \text,
    \\
    \rmat J_2 & =
    \py{my_latex(V["J"][1], -1, mat_str="array").replace('cc', "*{2}{X{3cm}}")}
    \text,
    \\
    \rmat J_3 & =
    \py{my_latex(V["J"][2], -1, mat_str="array").replace('cc', "*{2}{X{3cm}}")}
    \text.
  \end{align}
\end{myframe}
A formafüggvények globális és lokális koordináták szerinti parciális deriváltai
között a Jacobi-mát\-rix\-ok inverzei teremtenek kapcsolatot:
\begin{myframe}
  \begin{equation}
    \begin{bmatrix}
      \displaystyle \pdv{N_i}{x} \\[4mm]
      \displaystyle \pdv{N_i}{y} \\
    \end{bmatrix}
    =
    \underbrace{
      \begin{bmatrix}
        \displaystyle \pdv{\xi}{x} &
        \displaystyle \pdv{\eta}{x}  \\[4mm]
        \displaystyle \pdv{\xi}{y} &
        \displaystyle \pdv{\eta}{y}  \\
      \end{bmatrix}
    }_{\rmat J^{-1}}
    \begin{bmatrix}
      \displaystyle \pdv{N_i}{\xi}  \\[4mm]
      \displaystyle \pdv{N_i}{\eta} \\
    \end{bmatrix}
    \text.
  \end{equation}
\end{myframe}
Ezek paraméteresen:
\begin{myframe}
  \def\arraystretch{2.15}
  \begin{align}
    {\rmat J_1}^{-1} & =
    \py{my_latex(V["inv_J"][0], 4, mat_str="array").replace('cc', "*{2}{X{4cm}}")}
    \text,
    \\
    {\rmat J_2}^{-1} & =
    \py{my_latex(V["inv_J"][1], 4, mat_str="array").replace('cc', "*{2}{X{4cm}}")}
    \text,
    \\
    {\rmat J_3}^{-1} & =
    \py{my_latex(V["inv_J"][2], 4, mat_str="array").replace('cc', "*{2}{X{4cm}}")}
    \text.
  \end{align}
\end{myframe}

\subsection{Elemi merevségi mátrixok}

Tetszőleges síkbeli négyszögelem merevségi mátrixa az alábbi összefüggéssel
számítható:
\begin{myframe}
  \begin{equation}
    \rmat K_i
    = \int_{-1}^{+1} \int_{-1}^{+1}
    \rmat B^{\mathsf T}_i \,
    \rmat D_i \,
    \rmat B_i \,
    t \,
    \det \rmat J_i \,
    \mathrm d \xi \,
    \mathrm d \eta
    \text.
  \end{equation}
\end{myframe}

Az határozott integrál értéket $2 \times 2$-es Gauss-kvadratúra segítségével
fogjuk közelíteni:
\begin{myframe}
  \begin{equation}
    \rmat K_i \approx
    \sum_{i=1}^{2} \sum_{j=1}^{2}
    w_i \,
    w_j \,
    \rmat B^{\mathsf T}_i \,
    \rmat D_i \,
    \rmat B_i \,
    t \,
    \det \rmat J_i \,
    \text,
  \end{equation}
\end{myframe}

Az integrálási súlyok esetünkben: $w_1 = w_2 = 1$. A Gauss-pontok koordinátáit
pedig a \ref{fig:Gaussian}. ábra tartalmazza.
\begin{figure}[H]
  \centering
  \begin{tikzpicture}[thick, scale=3]
    \draw[draw=gray, fill=gray!10] (-1,-1) rectangle (1,1);
    \draw[ultra thick, -to] (-1.2,0) -- ++(2.5,0) node[below left] {$\xi$};
    \draw[ultra thick, -to] (0,-1.2) -- ++(0,2.5) node[below right] {$\eta$};

    % \node[dot] (t) at (-0.58, -.58) {}; \node[above right] at (t) {$(\xi_1; \eta_1)$}; \node[below left]  at (t) {$(-1/\sqrt{3}; -1/\sqrt{3})$};
    % \node[dot] (t) at (+0.58, -.58) {}; \node[above left]  at (t) {$(\xi_2; \eta_1)$}; \node[below right] at (t) {$( 1/\sqrt{3}; -1/\sqrt{3})$};
    % \node[dot] (t) at (+0.58, +.58) {}; \node[below left]  at (t) {$(\xi_2; \eta_2)$}; \node[above right] at (t) {$( 1/\sqrt{3};  1/\sqrt{3})$};
    % \node[dot] (t) at (-0.58, +.58) {}; \node[below right] at (t) {$(\xi_1; \eta_2)$}; \node[above left]  at (t) {$(-1/\sqrt{3};  1/\sqrt{3})$};

    \node[dot] (t) at (-0.58, -.58) {}; \node[below left,  xshift=4mm ] at (t) {$(-1/\sqrt{3}; -1/\sqrt{3}) = (\xi_1; \eta_1)$};
    \node[dot] (t) at (+0.58, -.58) {}; \node[below right, xshift=-4mm] at (t) {$(\xi_2; \eta_1) = ( 1/\sqrt{3}; -1/\sqrt{3})$};
    \node[dot] (t) at (+0.58, +.58) {}; \node[above right, xshift=-4mm] at (t) {$(\xi_2; \eta_2) = ( 1/\sqrt{3};  1/\sqrt{3})$};
    \node[dot] (t) at (-0.58, +.58) {}; \node[above left,  xshift=4mm ] at (t) {$(-1/\sqrt{3};  1/\sqrt{3}) = (\xi_1; \eta_2)$};
  \end{tikzpicture}
  \caption{Gauss-pontok szemléltelése}
  \label{fig:Gaussian}
\end{figure}

A képletben szerepel egy $\rmat B$ mátrix is, amely a formafüggvények $x$ és $y$
változók szerinti parciális deriváltjait tartalmazza az alábbi módon:
\begin{myframe}
  \newcommand{\nd}[2]{\displaystyle\pdv{N_{#1}}{#2}}
  \def\arraystretch{2}
  \begin{equation}
    \rmat B_i = \begin{bmatrix}
      \nd{1}{x} & 0         &
      \nd{2}{x} & 0         &
      \nd{3}{x} & 0         &
      \nd{4}{x} & 0
      \\
      0         & \nd{1}{y} &
      0         & \nd{2}{y} &
      0         & \nd{3}{y} &
      0         & \nd{4}{y}
      \\
      \nd{1}{y} & \nd{1}{x} &
      \nd{2}{y} & \nd{2}{x} &
      \nd{3}{y} & \nd{3}{x} &
      \nd{4}{y} & \nd{4}{x}
      \text.
    \end{bmatrix}
  \end{equation}
\end{myframe}

A $\rmat D$ mátrix pedig
\py{V["state"] == "SF" and "sík-feszültségi" or "sík-alakváltozási"} állapot
esetén az alábbi alakot veszi fel:
\begin{myframe}
  \newcommand{\sfa}[2]{\py{V["state"] == "SF" and r"#1" or r"#2"}}
  \def\arraystretch{1.15}
  \begin{equation}
    \rmat D_i =
    \sfa{\frac{E}{1 - \nu^2}}{\frac{E}{(1+\nu)(1-2\nu)}} \;
    \left[\begin{array}{*{3}{X{15mm}}}
        \sfa{1}{1-\nu} & \nu            & 0                         \\
        \nu            & \sfa{1}{1-\nu} & 0                         \\
        0              & 0              & \sfa{(1-\nu)}{(1-2\nu)}/2
      \end{array}\right]
    \text.
  \end{equation}
\end{myframe}

Az elemi merevségi mátrixok tehát numerikusan:
\begin{myframe}
  \begin{align}
    \siplaces{4}
    \sifix{}
    \rmat K_1 = \left[
      \scalebox{.75}{$\begin{array}{*{8}{X{1.55cm}}}
                            \pyc{print_matrix(V["K_i"][0], 1e-6, 1e-6)}
                          \end{array}$}
      \right]
    \cdot 10^6
    \\
    \siplaces{4}
    \sifix{}
    \rmat K_2 = \left[
      \scalebox{.75}{$\begin{array}{*{8}{X{1.55cm}}}
                            \pyc{print_matrix(V["K_i"][1], 1e-6, 1e-6)}
                          \end{array}$}
      \right]
    \cdot 10^6
    \\
    \siplaces{4}
    \sifix{}
    \rmat K_3 = \left[
      \scalebox{.75}{$\begin{array}{*{8}{X{1.55cm}}}
                            \pyc{print_matrix(V["K_i"][2], 1e-6, 1e-6)}
                          \end{array}$}
      \right]
    \cdot 10^6
  \end{align}
\end{myframe}

\subsection{A globális merevségi mátrix}

A globális merevségi mátrix meghatározásához írjuk fel az egyes elemekhez tartozó szabadsági foko-
kat mátrixosan:
\begin{myframe}
  \begin{equation}
    \rmat{DOF} =
    \py{latex(sp.Matrix(V["DOF"]), mat_str="array").replace("cccccccc", "*{8}{X{7mm}}")}
    \text.
  \end{equation}
\end{myframe}

A globális merevségi mátrix összeállításakor figyelnünk kell arra, hogy az adott
elem merevségi mátrixának megfelelő elemeit a hozzá tartozó szabadsági fokhoz
tartozó helyhez rendeljük hozzá. Ezt a \ref{fig:K-construction} ábra
szemlélteti.
\begin{figure}[ht]
  \centering
  \includestandalone{K-construction}
  \caption{A globális merevségi mátrix szemléletes felépítése}
  \label{fig:K-construction}
\end{figure}

A globális merevségi mátrix tehát az alábbi alakot veszi fel:
\begin{myframe}
  \begin{equation}
    \siplaces{4}
    \sifix{}
    \rmat K = \left[
      \scalebox{.45}{
        $\begin{array}{*{16}{X{1.45cm}}}
            \pyc{print_matrix(V["K"], 1e-6, 1e-6)}
          \end{array}$
      }
      \right]
    \cdot 10^6
    \text.
  \end{equation}
\end{myframe}

\subsection{A merevségi egyenlet}

Írjuk fel a globális elmozdulás-vektort. A görgős megtámasztás miatt a
(\py{P["F"]["x"][0]})-es és a (\py{P["F"]["x"][1]})-os pontokban az $x$,
az (\py{P["F"]["y"][0]})-ös és a (\py{P["F"]["y"][1]})-os pontokban pedig az
$y$ irányú elmozdulás gátolt, vagyis ezen elmozdulás-komponensek zérusak.
Ezek alapján a vektor paraméteresen:
\begin{myframe}
  \begin{equation}
    \rvec U = \py{latex(V["U_sym"].T)}^{\mathsf T}
    \text.
  \end{equation}
\end{myframe}

A görgős támasz miatt a gátolt irányokban kényszererők ébrednek, valamint
a megoszló terhelés, mint koncentrált erő is megjelenik a globális
terhelésvektorban, amely az alábbi alakot veszi fel:
\begin{myframe}
  \begin{equation}
    \rvec F = \py{my_latex(V["F_sym"].T, fold_short_frac=True)}^{\mathsf T}
    \text.
  \end{equation}
\end{myframe}

A rendszer merevségi egyenlete:
\begin{equation}
  \rmat K \rvec U = \rvec F
  \text.
\end{equation}

Fontos, hogy a merevségi egyenlet csak abban az esetben oldható meg mátrix
inverziós módszerrel, amennyiben $\rmat K$ mátrix reguláris. Mivel ez a
feltétel nem teljesül, ezért az egyenletrendszert kondenzálnunk kell.
A kondenzált merevségi egyenletrendszert úgy kapjuk meg, hogy a gátolt
szabadsági fokokhoz tartozó sorokat és oszlopokat töröljük az eredeti, globális
merevségi egyenletből:
\begin{myframe}
  \begin{equation}
    \sifix{}
    \siplaces{4}
    \underbrace{\left[
        \scalebox{.48}{
          $\begin{array}{*{12}{X{1.45cm}}}
              \pyc{print_matrix(V["K_kond"], 1e-16, 1e-6)}
            \end{array}$
        }
        \right]
      \cdot 10^6}_{\widehat{\rmat K}}
    \cdot
    \underbrace{\left[
    \scalebox{.48}{
    $\py{latex(V["U_sym_kond"], mat_delim="").replace('cc', "X{5mm}X{4.95mm}")}$
    }
    \right]}_{\widehat{\rvec U}}
    \siplaces{0}
    =
    \underbrace{\left[
        \scalebox{.48}{
          $\begin{array}{*{12}{X{1.45cm}}}
              \pyc{print_matrix(V["F_kond"])}
            \end{array}$
        }
        \right]}_{\widehat{\rvec F}}
    \text.
  \end{equation}
\end{myframe}

Mivel $\widehat{\rmat K}$ mátrix reguláris, ezért az egyenletrendszer
mátrix-invertálással megoldható, vagyis:
\begin{myframe}
  \begin{equation}
    \widehat{\rvec U}
    =
    \sifix{}
    \siplaces{4}
    \underbrace{\left[
        \scalebox{.5}{
          $\begin{array}{*{12}{X{1.45cm}}}
              \pyc{print_matrix(V["K_kond_inv"], 1e-16, 1e+6)}
            \end{array}$
        }
        \right]
      \cdot 10^{-6}}_{\widehat{\rmat K}^{-1}}
    \cdot
    \siplaces{0}
    \underbrace{\left[
        \scalebox{.5}{
          $\begin{array}{*{12}{X{1.45cm}}}
              \pyc{print_matrix(V["F_kond"])}
            \end{array}$
        }
        \right]}_{\widehat{\rvec F}}
    \text.
  \end{equation}
\end{myframe}

Az egyenlet megoldása numerikusan:
\begin{myframe}
  \begin{equation}
    \siplaces{6}
    \widehat{\rvec U} = \left[
      \scalebox{.65}{
        $\begin{array}{c}
            \pyc{print_matrix(V["U_kond"])}
          \end{array}$
      }
      \right]
    \text{mm}
    \text.
  \end{equation}
\end{myframe}

A globális elmozdulásvektorba visszahelyettesítve:
\begin{myframe}
  \begin{equation}
    \siplaces{6}
    \rvec U = \left[
      \scalebox{.65}{
        $\begin{array}{c}
            \pyc{print_matrix(V["U_calc"])}
          \end{array}$
      }
      \right]
    \text{mm}
    \text.
  \end{equation}
\end{myframe}

Az eredő elmozdulások a Pitagorasz-tétel alapján számíthatóak:
\begin{myframe}
  \begin{equation}
    \Delta_i = \sqrt{{U_i}^2 + {V_i}^2}
    \text.
  \end{equation}
\end{myframe}

Numerikusan:
\begin{myframe}
  \begin{equation}
    \rvec{\Delta} = \begin{bmatrix}
      \Delta_1 \\
      \Delta_2 \\
      \Delta_3 \\
      \Delta_4 \\
      \Delta_5 \\
      \Delta_6 \\
      \Delta_7 \\
      \Delta_8 \\
    \end{bmatrix}
    =
    \siplaces{6}
    \begin{bmatrix}
      \pyc{print_matrix(V["Delta_mm"])}
    \end{bmatrix}
    \text{mm}
    =
    \siplaces{3}
    \begin{bmatrix}
      \pyc{print_matrix(V["Delta_um"])}
    \end{bmatrix}
    \upmu\text{m}
    \text.
  \end{equation}
\end{myframe}
